\documentclass{article}


% figures
\usepackage{subcaption}
\usepackage{graphicx}

% citations
%\usepackage[round]{natbib}   % omit 'round' option if you prefer square brackets
%\bibliographystyle{plainnat}
\bibliographystyle{abbrv}
%\bibliographystyle{acm}
%\bibliographystyle{alpha}
%\bibliographystyle{apalike}
%\bibliographystyle{ieeetr}
%\bibliographystyle{plain}
%\bibliographystyle{siam}
%\bibliographystyle{unsrt}

\begin{document}
\title{Problem Set 1}
\author{Xinxin Zhang}
\date{\today}

\maketitle

I am still exploring my interest area. I am currently working on a project which tests the effect of securitization on investors. Securitization is the financial practice of pooling various types of assets, such as mortgages, auto loans or credit card debt obligations and selling their related cash flows to third party investors as securities. Investors are repaid from the principal and interest cash flows collected from the underlying portfolio and redistributed through the capital structure of the new financing. 
Securitization provides many advantages to investors. Firstly, investors have opportunity to potentially earn a higher rate of return from structured securities than corporate bonds. Due to the stringent requirements for corporations (for example) to attain high ratings, there is a dearth of highly rated entities that exist. Securitizations, however, allow for the creation of large quantities of AAA, AA or A rated bonds, and risk averse institutional investors, or investors that are required to invest in only highly rated assets, have access to a larger pool of investment options. Secondly, investors can earn profits from portfolio diversification.
Thirdly, investors are immune from credit risk from the parent entity. Since the assets that are securitized are isolated (at least in theory) from the assets of the originating entity, under securitization it may be possible for the securitization to receive a higher credit rating than the "parent," because the underlying risks are different. For example, a small bank may be considered more risky than the mortgage loans it makes to its customers; were the mortgage loans to remain with the bank, the borrowers may effectively be paying higher interest (or, just as likely, the bank would be paying higher interest to its creditors, and hence less profitable).
From advantages mentioned above, we can predict that investors should earn excess returns from structured securities than corporate bonds. I examine excess returns of securitized bonds and corporate bonds and want to test whether securitized bonds provide a better excess return than corporate bonds.

I read lots of literature on securitizaion. One of them is \cite{han2015corporate}.

I use CAPM model to derive excess return of bonds, which is $\alpha$
\begin{equation} 
\ R_i-R_f=\alpha+\beta\cdot (R_m-R_f)
\end{equation}
Figure 1 shows quarterly return of different types of bonds (corporate bonds, ABS, MBS, and CLO)
\begin{figure}[h]
\centering
\includegraphics[scale=0.5]{picture1.png}
\end{figure}

\bibliography{references} 
\end{document}

